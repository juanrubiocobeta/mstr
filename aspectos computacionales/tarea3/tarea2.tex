\documentclass{article}
\usepackage{amsmath}
\usepackage{amsfonts}
\usepackage{amssymb}
\usepackage[utf8]{inputenc}
\usepackage{fancyhdr}
\usepackage{geometry}
\usepackage[hidelinks]{hyperref}

\geometry{top=3cm, bottom=3cm, left=2cm, right=2cm}

\pagestyle{fancy}
\fancyhf{}
\fancyhead[L]{Ejercicios Propuestos 2}
\fancyhead[C]{Juan Rubio Cobeta}
\fancyhead[R]{\today}

\title{Ejercicios Propuestos}
\author{Juan Rubio Cobeta}
\date{\today}

\begin{document}
\maketitle

\tableofcontents

\newpage

\maketitle

\section{Apartado a)}

Queremos estimar la tasa de desempleo $R$ en una región, definida como el cociente entre el total de personas desempleadas ($Y$) y el total de personas activas ($X$).

Para ello se selecciona la muestra mediante un muestreo por conglomerados en dos etapas:

\vspace{0.3cm}

\noindent \textbf{Etapa 1 (PPT):} 60 municipios (UPM) seleccionados con probabilidad proporcional a su tamaño.

\noindent \textbf{Etapa 2 (MAS):} 10 hogares por municipio (600 hogares en total).

\vspace{0.3cm}

Dividimos las 60 UPM en 4 grupos (15 municipios por grupo). Los datos obtenidos son los siguientes:

\begin{center}
\begin{tabular}{|c|c|c|}
\hline
Grupo & Total inactivos & Total edad de trabajar \\
\hline
1 & 210 & 420 \\
2 & 234 & 450 \\
3 & 198 & 440 \\
4 & 225 & 459 \\
\hline
\end{tabular}
\end{center}

1) Estimar la tasa de desempleo $R$ y su varianza mediante la técnica de los grupos aleatorios

\subsection{Solución}

Para estimar la tasa de desempleo $R$ y su varianza mediante la técnica de los grupos aleatorios, consideramos cada uno de los $k=4$ grupos como réplicas independientes del diseño.

\subsection*{1. Cálculo de las estimaciones por grupo ($\hat{R}_t$)}

Calculamos la tasa estimada $\hat{R}_t = \frac{Y_t}{X_t}$ para cada grupo $t$, donde $Y_t$ es el total de inactivos y $X_t$ es el total en edad de trabajar:

\begin{itemize}
    \item \textbf{Grupo 1:} 
    \[ \hat{R}_1 = \frac{210}{420} = 0.5000 \]
    \item \textbf{Grupo 2:} 
    \[ \hat{R}_2 = \frac{234}{450} = 0.5200 \]
    \item \textbf{Grupo 3:} 
    \[ \hat{R}_3 = \frac{198}{440} = 0.4500 \]
    \item \textbf{Grupo 4:} 
    \[ \hat{R}_4 = \frac{225}{459} \approx 0.4902 \]
\end{itemize}

\subsection*{2. Estimación de la tasa global ($\hat{R}_{gi}$)}

El estimador puntual mediante grupos aleatorios es el promedio simple de las estimaciones de cada grupo:

\[
\hat{R}_{gi} = \frac{1}{k} \sum_{t=1}^{k} \hat{R}_t
\]

Sustituyendo los valores obtenidos ($k=4$):

\[
\hat{R}_{gi} = \frac{0.5000 + 0.5200 + 0.4500 + 0.4902}{4} = \frac{1.9602}{4} = 0.49005
\]

Por tanto, la tasa de desempleo estimada es del 49,00\%.

\subsection*{3. Estimación de la varianza ($\hat{V}(\hat{R}_{gi})$)}

La fórmula para la varianza estimada del estimador global mediante grupos aleatorios es:

\[
\hat{V}(\hat{R}_{gi}) = \frac{1}{k(k-1)} \sum_{t=1}^{k} (\hat{R}_t - \hat{R}_{gi})^2
\]

Calculamos primero la suma de cuadrados de las desviaciones respecto a la media global ($0.49005$):

\begin{align*}
\sum_{t=1}^{4} (\hat{R}_t - \hat{R}_{gi})^2 &= (0.5000 - 0.49005)^2 + (0.5200 - 0.49005)^2 \\
&\quad + (0.4500 - 0.49005)^2 + (0.4902 - 0.49005)^2 \\
&= (0.00995)^2 + (0.02995)^2 + (-0.04005)^2 + (0.00015)^2 \\
&\approx 0.000099 + 0.000897 + 0.001604 + 0.000000 \\
&= 0.002600
\end{align*}

Finalmente, aplicamos el factor $\frac{1}{k(k-1)} = \frac{1}{4(3)} = \frac{1}{12}$:

\[
\hat{V}(\hat{R}_{gi}) = \frac{1}{12} \cdot 0.002600 \approx 0.000217
\]

\vspace{0.5cm}

\noindent \fbox{\begin{minipage}{\textwidth}
\textbf{Resumen de resultados:}
\begin{itemize}
    \item Estimación de la tasa de desempleo: $\hat{R} = 49.00\%$
    \item Varianza estimada: $\hat{V}(\hat{R}) = 0.000217$
\end{itemize}
\end{minipage}}

\newpage

\section{Apartado b)}

Un establecimiento recreativo cobra la admisión por automóvil en vez de por persona, y un ejecutivo de la empresa quiere estimar el número medio de personas por automóvil para un día particular. Sabe por experiencia que ese día entrarán alrededor de 400 automóviles y quiere muestrear 80 de ellos. Para obtener un estimador de la varianza utiliza un muestreo sistemático replicado con 10 muestras de 8 vehículos cada una. Los datos obtenidos se muestran en la tabla siguiente:

\vspace{0.3cm}

\begin{center}
\begin{tabular}{|l|c|c|c|c|c|c|c|c|}
\hline
m. 1 & 2 & 3 & 5 & 1 & 4 & 5 & 3 & 6 \\
m. 2 & 5 & 2 & 3 & 2 & 1 & 1 & 2 & 3 \\
m. 3 & 3 & 3 & 1 & 5 & 3 & 4 & 2 & 2 \\
m. 4 & 2 & 2 & 1 & 5 & 5 & 2 & 3 & 1 \\
m. 5 & 3 & 4 & 2 & 3 & 1 & 1 & 5 & 2 \\
m. 6 & 2 & 4 & 5 & 1 & 1 & 3 & 2 & 3 \\
m. 7 & 3 & 2 & 2 & 3 & 1 & 1 & 4 & 2 \\
m. 8 & 4 & 3 & 2 & 4 & 4 & 2 & 1 & 1 \\
m. 9 & 2 & 1 & 2 & 4 & 2 & 2 & 4 & 3 \\
m. 10 & 4 & 5 & 2 & 2 & 1 & 4 & 2 & 3 \\
\hline
\end{tabular}
\end{center}

\vspace{0.3cm}

Estimar el número medio de personas por automóvil mediante un intervalo de confianza al 95\% usando el método de los grupos aleatorios. Establecer el error de muestreo.

\subsection{Solución}

Para estimar el número medio de personas por automóvil ($\bar{X}$) y su varianza, utilizamos la técnica de los grupos aleatorios con $k=10$ muestras sistemáticas replicadas (grupos), cada una de tamaño $m=8$.

\subsection*{1. Cálculo de las estimaciones por grupo ($\bar{x}_t$)}

Calculamos la media muestral $\bar{x}_t = \frac{1}{m} \sum x_{tj}$ para cada uno de los 10 grupos:

\begin{itemize}
    \item \textbf{m.1:} $\bar{x}_1 = (2+3+5+1+4+5+3+6)/8 = 3.625$
    \item \textbf{m.2:} $\bar{x}_2 = (5+2+3+2+1+1+2+3)/8 = 2.375$
    \item \textbf{m.3:} $\bar{x}_3 = (3+3+1+5+3+4+2+2)/8 = 2.875$
    \item \textbf{m.4:} $\bar{x}_4 = (2+2+1+5+5+2+3+1)/8 = 2.625$
    \item \textbf{m.5:} $\bar{x}_5 = (3+4+2+3+1+1+5+2)/8 = 2.625$
    \item \textbf{m.6:} $\bar{x}_6 = (2+4+5+1+1+3+2+3)/8 = 2.625$
    \item \textbf{m.7:} $\bar{x}_7 = (3+2+2+3+1+1+4+2)/8 = 2.250$
    \item \textbf{m.8:} $\bar{x}_8 = (4+3+2+4+4+2+1+1)/8 = 2.625$
    \item \textbf{m.9:} $\bar{x}_9 = (2+1+2+4+2+2+4+3)/8 = 2.500$
    \item \textbf{m.10:} $\bar{x}_{10} = (4+5+2+2+1+4+2+3)/8 = 2.875$
\end{itemize}

\subsection*{2. Estimación de la media global ($\hat{\bar{X}}$)}

El estimador puntual global es el promedio de las medias de los grupos:

\[
\hat{\bar{X}} = \frac{1}{k} \sum_{t=1}^{k} \bar{x}_t = \frac{27.000}{10} = 2.70
\]

Se estima que el número medio de personas por automóvil es 2.70.

\subsection*{3. Estimación de la varianza ($\hat{V}(\hat{\bar{X}})$)}

La varianza del estimador global mediante grupos aleatorios se calcula como:

\[
\hat{V}(\hat{\bar{X}}) = \frac{1}{k(k-1)} \sum_{t=1}^{k} (\bar{x}_t - \hat{\bar{X}})^2
\]

Primero obtenemos la suma de cuadrados de las desviaciones respecto a la media global ($2.70$):
\[
\sum_{t=1}^{10} (\bar{x}_t - 2.70)^2 = (0.925)^2 + (-0.325)^2 + \dots + (0.175)^2 = 1.2875
\]

Aplicamos el factor $\frac{1}{k(k-1)} = \frac{1}{90}$:

\[
\hat{V}(\hat{\bar{X}}) = \frac{1.2875}{90} \approx 0.014306
\]

El error estándar estimado ($\widehat{SE}$) es:
\[
\widehat{SE} = \sqrt{0.014306} \approx 0.1196
\]

\subsection*{4. Intervalo de Confianza y Error de Muestreo}

Para un nivel de confianza del 95\%, utilizamos la distribución t-Student con $k-1 = 9$ grados de libertad. El valor crítico es $t_{0.025, 9} \approx 2.262$.

El error de muestreo (margen de error) es:
\[
E = t \cdot \widehat{SE} = 2.262 \cdot 0.1196 \approx 0.2706
\]

El intervalo de confianza es:
\[
IC_{95\%} = 2.70 \pm 0.2706 = [2.429, 2.971]
\]

\vspace{0.5cm}

\noindent \fbox{\begin{minipage}{\textwidth}
\textbf{Resumen de resultados:}
\begin{itemize}
    \item Estimación media: 2.70 personas/automóvil.
    \item Error de muestreo: $\pm 0.27$ (al 95\% de confianza).
    \item Intervalo: Entre 2.43 y 2.97 personas por automóvil.
\end{itemize}
\end{minipage}}

\end{document}